\documentclass[12pt]{article}
\usepackage[utf8]{inputenc}
\usepackage[T1]{fontenc}
\usepackage{graphicx}
\usepackage{xcolor}
\usepackage{setspace}
\usepackage{pdflscape}
\usepackage{subfig}
\onehalfspacing
\input{defs}
\usepackage{adate/authordate1-4}
\usepackage[toc, page]{appendix}




%%%%%%%%%%%%%%%
% Title Page
\title{BUZZ LLC}
\author{DU ID: nbgt74 \\(1st Year Computer Science Student)}
\date{\today}
\summary{Business plan based on the idea of Seminar group 5-1 
        \\(word count: 3000)}
%%%%%%%%%%%%%%%

\begin{document}


\maketitle


\tableofcontents
\listoffigures
\clearpage


\section{Executive Summary}
 


\newpage

\section{The Business}

BUZZ's core value proposition is based on Snapchat’s model of self-destructing messages. The short life-span of the messages allows users to be more spontaneous: they don’t have to worry about following standard social conventions as there is no pressure to have a “conversation”, helping create the fun, simple environment that makes BUZZ stand out from other, standard messaging apps.

\begin{figure}[!h]
    \centering
    \includegraphics[width=\textwidth]{Figures/NV_Type.jpg}
    \caption{Our New Venture Typology \cite{Burns2014} }
    \label{fig:nv type}
\end{figure}

The Buzz LLC will operate under a freemium model(Osterwalder \& Pigneur, 2010, Figure \ref{fig:bmc}). Making the app free-to-download helps to maximise on potential downloads, as about 66\% of all mobile users never pay for apps \cite{eMarketer2015} . This aids the firm's primary objective of growing a big enough user base to ensure profitable amounts of daily activity.


\begin{figure}
    \centering
    \includegraphics[width = \textwidth]{Figures/BMC.jpg}
    \caption{Business Model Canvas, \cite{Osterwalder2010}}
    \label{fig:bmc}
\end{figure}

\subsection{Making The App}

A key element to success is building a well-designed app that functions properly. Though the functionality is quite trivial at the user interface level, it takes a lot of craft to design a backend that scales well.
Therefore, despite having a member within the team who has some experience in software design, we will delegate this task to a Professional Software Development firm. This ensures BUZZ makes a good first-impression on the market and lays a solid foundation for any subsequent updates to add more functionality to the app.

Only the core structure of the app will be externally developed; the remainder will be done in-house. This involves getting the sounds, setting up the advertising network, and finally, setting up the server, which will store all the user information and facilitate peer-to-peer communication:

\begin{itemize}
    \item The sounds will be downloaded from various free sound databases online then edited, if necessary, using open-source sound editing software.
    \item The advertisement content will retrieved from AdMob, a massive ad database ran by Google which can be accessed for free after going through a simple registration process. On average, Google will pay us about £0.01 per click-thru.
    \item  A virtual server will be used in order to avoid the initial huge capital cost of buying a physical one, with the web-host charging about £9 per month. This fee is likely to increase as our user base grows beyond 20,000.
\end{itemize}


\subsection{The Team and Key Activities}

The core team consists of three members: Ucizi Mafeni, 1st year Computer Science Student; Karan Argawal, 1st Year Economics Student; and Oliver Robley, also 1st Year Economics Student. They will be the primary shareholders, jointly holding the dominating stake, and also three of the four members on the Board of Directors. The firm's angel investor will also have equity in the firm and will most likely be a non-executive director, though we will draw upon his expertise in the formation of the company.
%MAKE A Table:
The day-to-day activities will be split as follows:
\begin{itemize}
    \item As the creator of Buzz, Karan, CEO and founder, will be responsible for the overall vision and direction of the company. Aside from this, his more active role will mainly deal with gauging customer satisfaction. This will involve looking through and analysing our customer reviews on the Application Stores and any feedback emails received through our website.
    \item Oliver,CMO, CFO and co-founder, will oversee the execution of our various marketing strategies. He will also be in charge of the financial bookkeeping, which should be quite trivial within the first few years of operation.
    \item Ucizi, CTO and co-founder, will oversee all the technical aspects of our app. He'll personally be responsible for the building of the website due to his experience in HTML and CSS. He'll'll also monitor the back-end of the app i.e. the user and sound databases as well the functionality of our virtual server and the necessary upgrades that will be needed as our user base grows.
\end{itemize}

In addition, there may also be a need to hire more experienced workers (i.e. software developers, audio producers) on a part-time basis for the more complicated tasks. Once the company becomes more profitable, a full-time development and marketing team will be brought in to ensure BUZZ scales successfuly.

\section{Our Place in the Market}

\subsection{Macro Environment}

\begin{figure}
    \centering
    \includegraphics[width = \textwidth]{Figures/PESTLE.jpg}
    \caption{PESTLE analysis, Information References: \cite{PWC2015}, \cite{eMarketer}, \cite{HM_Treasury2015}}
    \label{fig:pestle}
\end{figure}

\subsubsection*{Political and Economic factors}
The UK economy has been showing good recovery signs from the financial crisis in 2009. With this still fresh on most of the public's mind, Political parties are urged to introduce policies that encourage economic growth and boost consumer confidence. Many of these policies work to the advantage of small businesses (i.e. tax breaks, low interest rates) and are likely to remain the same till 2018. %This gives us plenty of time to grow to a point at which it can more capably handle any unfavourable policy changes.
This gives us plenty of time to grow our business, so we will better equipped to handle any unfavourable policy changes.

\subsubsection*{Social and Technological factors}
Smart phones these days have become a necessity: almost everyone has one, and almost no one can live without one. Though there are prospects of emerging technologies (wearable computers), they are still in the early phases of development and it will take a while before they replace the role of smartphones. As such there is no expected decline in the smartphone industry (which would basically put us out of business)

iOS and Android, the two platforms on which BUZZ will be launched, currently dominate the smartphone market, and with their continuous innovations and development, they’re expected stay on top for the foreseeable future. As such, there is little fear of either markets declining.

\subsection{Industry/Sector}

\begin{figure}
    \centering
    \includegraphics[width = \textwidth]{Figures/PORTERS.jpg}
    \caption{Porter’s Five Forces \cite{Burns2014}}
    \label{fig:porters}
\end{figure}

Though the cost of making an app relatively low, as all one really needs is a PC and the ability to code, anyone looking to monetise their app faces the significant challenge of having to market their product, since most new apps are practically invisible to the public eye. Getting noticed on the App Store is especially difficult: the apps most likely to be downloaded are usually at the top of the search list, and the apps at the top of the search are usually the ones with millions of downloads.
 
Most new apps, which usually don't have a huge advertising budget, rely heavily on social media, word of mouth, peer to peer sharing (e.g. app invite messages) and may even resort to slightly unconventional methods. However, its all worth it because once prospective users discover the app, they lose nothing by downloading it and even if it's deleted soon after, the burst of activity contributes towards your visibility on the app store. Therefore once the app is developed, a lot of the company's activity will be centred on marketing and promotion.

\subsection{Competitors}

\begin{figure}
    \centering
    \includegraphics[width = \textwidth]{Figures/COMPETITORS.jpg}
    \caption{Our main competitors, References: \cite{Shontell2015}, \cite{Facebookn.d.}, \cite{WhatsAppn.d.}
}
    \label{fig:competitors}
\end{figure}

Any app that is a messaging app, of which are several of, can be considered a competitor since they will all be listed under the same category as BUZZ app on the Store. As it is most similar in nature, Snapchat is probably our most direst competitor. However the other major messaging apps e.g. Viber, Facebook messenger, and Whatsapp, are potential threats as they already have a large worldwide user base, and could potentially add the features that make BUZZ unique to their own respective apps.

\subsection{Our Target Market}

\begin{figure}
    \centering
    \includegraphics[width = \textwidth]{Figures/TARGET_MARKET.jpg}
    \caption{TAM SAM SOM diagram}
    \label{fig:target market}
\end{figure}

Critical Mass is a commonly used term in the app industry. It’s essentially “the minimum size of the user base at which enough number of producers and consumers exist to spark transactions sustainably" \cite{Choudaryn.d.}. The company's current goal is to hit an estimated critical mass of 50,000 users within the first two years of operation (See Figure \ref{fig:target market}. We believe this objective should be achievable within our targeted market demographic (UK University students). 

\section{Sales and Marketing}
Before hitting critical mass, a majority of new users will be gained through advertising and promotion. In order to overcome having a small user base, we will focus on micro-marketing: promoting within specific geographic areas so our user base is relevant. This increases the likelihood of having multiple users within the same social group, resulting in more interaction within the app. We'll initially concentrate on promoting the app within Durham, then move on to other universities in the north-east, and other major universities.

\subsection{Marketing Mix}

\subsubsection*{Price}

In-app purchases will serve the purpose of raising enough revenue to remain solvent while waiting for the user base to grow large enough for the native advertising to become profitable. Following, an online survey we carried out (See Appendix 1), it was decided that the ad-free version would be priced at £0.99, and the premium sounds would be priced at approximately £0.30 per sound. We believe these prices would bring in the most sales revenue as it is roughly the price most respondents were willing to pay.

\subsubsection*{Promotion}\label{sec:Promotion}
We will initially promote use within our own university - a strategy employed by many social networking/messaging apps in their early stages (e.g. Facebook at Harvard, YikYak in Georgia). Once it becomes popular within Durham, we'll proceed to promoting it in neighbouring Universities and soon enough to the rest of the country.
 
Our main strategies will include  Email-marketing, Social media pages (twitter and Facebook), Sponsoring events within the University and Running Promotional Tours i.e. handing out free merchandise and hiring campus reps to help promote the app.
 
For advertising, we'll use Google AdWords, which brings up a link to our app every time someone types in one of our selected keywords on Google. It's very cost-effective as we only pay for each click-thru and we have of additional option of targeting specific geographic locations, which aids in adding relevant users to our network.

\subsubsection*{Place (Product Distribution)}
BUZZ will made available on the Google PlayStore and the Apple AppStore. In addition , we'll place links to both download locations on all our Social Media pages as well as the App Website. Furthermore, an app invite function will be embedded within the app, which allows users to send a download link to contacts who don't currently have the app.

\subsubsection*{Product}
The core concept behind the design of our app is "Friction-Free": upon opening the app, users should be able to effortlessly listen to and send sound messages. This will be achieved through a simple designed navigation menu, from where users can see a list of their new messages and contacts they most recently interacted with. The sounds will be grouped into categories, with the first one containing their most frequently used sounds, which they can send with a single tap. (Refer to the sample design on Figure \ref{fig:ui})

\begin{figure}[!h]
    \centering
    \subfloat[home screen]{
    \includegraphics[width=0.4\textwidth]{Figures/SCREEN1}
    }
    \subfloat[message screen]{
    \includegraphics[width=0.4\textwidth]{Figures/SCREEN2}
    }
    \caption{Sample User Interface Layout}
    \label{fig:ui}
\end{figure}

\section{Financial Analysis}

\subsection{Break Even analysis}
Since the models used to estimate ad revenues, in comparison to revenues from purchases are somewhat incompatible (i.e. Ad's depend on active users whereas Purchases are more dependent on new users),separate Break-Even charts were made to depict these two equally important aspects of our revenue model.

There are only two main costs directly related to our product: The AppStore subscription fee (to keep our app on the store) and the cost of renting out Virtual Server space (to store all user details and facilitate messaging). Both are fixed in the short-term (i.e for a fixed number of users) but renting out Server space is variable in the long-run as our user base grows.

\subsubsection*{Gross Profit Break Even Point}
Within our first year, we'll need about 3 ad-free purchases or 11 sound purchases per-week to cover direct costs. Note variable costs are 0 as the cost of renting server space is fixed at this scale. Based on our Market Research, we expect about 30\% of new users to buy the ad-free version and 20\% to buy sounds. Therefore we'd need roughly 11 new users per week.

Alternatively, we'd need about 3700 active users per week to cover direct costs solely from ads. This is under the assumption that 10\% of our active users will click at least 1 ad per week, since most ads often have a low click-thru rate.

\begin{figure}[p]
    \centering
    \includegraphics[width = 0.8\textwidth]{Figures/BE1.jpg}
    \caption{Gross Profit Break Even Point from in-app purchases}
    \label{fig:BE_GP_Purchases}
\end{figure}

\begin{figure}[p]
    \centering
    \includegraphics[width = 0.8\textwidth]{Figures/BE2.jpg}
    \caption{Gross Profit Break Even Point from advertising}
    \label{fig:BE_GP_Ads}
\end{figure}

\subsubsection*{Net Profit Break Even Point}
Though estimated is cost of promoting our product is actually an indirect cost, including it in our total costs gives us a more realistic target of the sales we need to break even.Based on the combined expected marginal earnings from sounds and ad-free purchases (under the assumptions stated before) we'd expect to break even at a growth rate of about 88 new customers per week.On the other hand, we'd need 87,000 weekly active users to break even solely from ad revenue.

It is important to note that the break-even point from purchases is only worth considering in the short-run, as user growth rates will inevitably slow down. On the other hand, the break-even point from advertising revenues gives us an indicator of how big a user base we’ll need to sustain long-term profitability.

\begin{figure}[p]
    \centering
    \includegraphics[width = 0.8\textwidth]{Figures/BE3.jpg}
    \caption{Break Even Point from in-app purchases with promotion costs}
    \label{fig:BE_NP_Purchases}
\end{figure}

\begin{figure}[p]
    \centering
    \includegraphics[width = 0.8\textwidth]{Figures/BE4.jpg}
    \caption{Break Even Point from ads with promotion costs}
    \label{fig:BE_NP_Ads}
\end{figure}

\subsection{Cash Flow Forecast}
We're expected to just about maintain positive cash flow throughout the first year (See Figure \ref{fig:CF16}). Most of the funds will be spent on app development, which is likely to take up the first two months of operation. The remainder will mostly be used for promotional activities (Refer to section \ref{sec:Promotion})

However this forecast was only feasible after Management collectively decided to forgo their salaries. As the management team is mostly comprised of students, who are still dependents, they should be able off the support of their parents.

Though the second year's forecasts (See Figure \ref{CF17} are purely estimates, they give a good picture of how the financial period may go if all goes according to plan. Our revenue model was modelled user the assumption that user growth is a litte slow in the first year, then exponentiates through the second year once BUZZ's promotional campaigns become more successful at raising product awareness. This is why there is significant increase in revenue during the second year. For a more detailed description of estimated user growth figure, refer to Appendix 2)

\begin{landscape}

    \begin{figure}
        \centering
        \includegraphics[width = 1.4\textwidth]{Financials/CF_16.png}
        \caption{Cash Flow Forecast, 1st year}
        \label{fig:CF16}
    \end{figure}

    \begin{figure}
        \centering
        \includegraphics[width = 1.4\textwidth]{Financials/CF_17.png}
        \caption{Cash Flow Forecast, 2nd year}
        \label{fig:CF17}
    \end{figure}
    
\end{landscape}

\begin{figure}[!hp]
    \centering
    \includegraphics[width = \textwidth]{Financials/PL_16.jpg}
    \caption{Profit and Loss, 1st year}
    \label{fig:PL16}
\end{figure}

\begin{figure}[!hp]
    \centering
    \includegraphics[width = \textwidth]{Financials/PL_17.jpg}
    \caption{Profit and Loss, 2nd year}
    \label{fig:PL17}
\end{figure}

\begin{figure}[!hp]
    \centering
    \includegraphics[width = \textwidth]{Financials/BS_16.jpg}
    \caption{Balance Sheet, 1st year}
    \label{fig:BS16}
\end{figure}

\begin{figure}[!hp]
    \centering
    \includegraphics[width = \textwidth]{Financials/BS_17.jpg}
    \caption{Balance Sheet, 2nd year}
    \label{fig:BS17}
\end{figure}

\subsection{Profit \& Loss and Balance Sheet}
We expect to make a loss in the first year, mostly because of the cost of developing the app, though it’s highly likely we’ll break even the second year once sales have picked up.
The company will able to remain quite liquid as our only real fixed assets are our laptops. Most of our other required assets are either free or rented out. Hopefully, there’ll be no need to take out any long-term loans, so we should be able to maintain a good current ratio throughout the first two years, and therefore remain fully operational.



\section{Risk Assessment}

Our potential risks are highlighted in Figure \ref{fig:risks}. The copyright infringement lawsuit is probably the most hazardous risk. As a start-up, we hardly have enough resources to go to court and so our best bet is to play it safe and be diligent: we will only use trusted sound databases, which contain the licensing information of the sounds.  
 
The possibility of someone copying our idea also poses a great threat. We’ll look to get patents to protect any features that make our app stand out, and possibly also try to register key elements of user interface as designs. However, since patents are quite costly to obtain in terms of both time and money, we’ll only begin the process when it becomes clear that our app has the potential to a huge player in the social media market.

There is also a chance the University may seize BUZZ and all its contents, since legally our intellectual property belongs to them. To avoid any legal issues in the future, we'll look into drafting a contractual agreement between both parties. This could ultimately work to our advantage, we could draw on the experience the University has to help speed up the patent application process.

\begin{figure}[!hb]
    \centering
    \includegraphics[width = 0.7\textwidth]{Figures/RISK_MATRIX.jpg}
    \caption{
    Risk Matrix used to categorise risks in Figure \ref{fig:risks}}
    \label{fig:riskMatrix}
\end{figure}

\begin{figure}[!hp]
    \centering
    \includegraphics[width = 0.9\textwidth]{Figures/RISKS.jpg}
    \caption{Potential risks and mitigating actions (See \ref{fig:riskMatrix} for risk category matrix)}
    \label{fig:risks}
\end{figure}

\section{Long-Term Goals}
If BUZZ successfully hits critical mass in the first 24 months and there’s still potential for more growth, we’ll look to build on our success by hiring a full-time, in-house development team to help with new ideas to help improve the app. This will also mean moving into a formal office space. We’ll probably need to buy a physical server as well, as it’ll be more cost effective and give us more flexibility on how we handle our information databases. All this will be quite costly and most likely require more external funding, though our established market presence will make it easier to attract new investors.

\section{Exit Strategies}

Being A limited company offers a lot of flexibility in terms of exit due to the separation between ownership and management. Once BUZZ hits critical mass, it's likely that the current managerial team, which consists of the owners, will most likely be replaced with more experienced individuals who will be able to handle the growth during this stage. After this, it will be up to them to decide whether or not to sell their stake in the company. This shouldn't be too difficult as BUZZ's established market presence will make it easier to attract accredited investors.

However, if it becomes clear that the app cannot scale, Buzz llc will probably end up being acquired by a larger messaging company, since BUZZ is the only product and and the owners have expressed no intention of releasing any more apps.


\newpage
\bibliography{references}
\bibliographystyle{authordate1}

\newpage

\renewcommand*\appendixpagename{\headingfont\HUGE\bfseries\scshape\color{color1} Appendices}





\end{document}          
